% Created 2022-01-18 mar. 10:22
% Intended LaTeX compiler: pdflatex
\documentclass[11pt]{article}
\usepackage[utf8]{inputenc}
\usepackage[T1]{fontenc}
\usepackage{graphicx}
\usepackage{longtable}
\usepackage{wrapfig}
\usepackage{rotating}
\usepackage[normalem]{ulem}
\usepackage{amsmath}
\usepackage{amssymb}
\usepackage{capt-of}
\usepackage{hyperref}
\author{Victor}
\date{\today}
\title{2022-01-14}
\hypersetup{
 pdfauthor={Victor},
 pdftitle={2022-01-14},
 pdfkeywords={},
 pdfsubject={},
 pdfcreator={Emacs 29.0.50 (Org mode 9.5)}, 
 pdflang={English}}
\begin{document}

\maketitle
\tableofcontents

\section{Réunion Arthur/Olivier}
\label{sec:org45dd951}
\subsection{Autoencoders with Invertible maps}
\label{sec:orgefb5202}

\begin{align}
x &= \phi^{-1} \circ \phi(x) \\
&= \phi^{-1} \left(U_m U_m^T \phi(x) + U_{\bot}U_{\bot}^T \phi(x)\right)
\end{align}
Where \(U=\left[U_m \quad U_{\bot}\right]\) is unitary
\begin{align}
E(x) & = U_m^T \phi(x) \\
D(z) &= \phi^{-1}(U_m z + U_{\bot}z_0)
\end{align}
Construction of \(\phi\) ? ResNet, RevNet ?

\section{Arthur: GN}
\label{sec:orgf9e3559}
\subsection{3DVar: Cost function, Gradient and Hessian}
\label{sec:orgf748139}
Revoir \cite{gratton_approximate_2007}

\begin{align}
J(x) &= \frac{1}{2}\|(\mathcal{H}\circ \mathcal{M})(x) - y \|^2_{R} +\frac{1}{2} \|x - x_b \|^2_{B}\\
&= \frac{1}{2}\|G(x) - y \|^2_{R} + \frac{1}{2}\|x - x_b \|^2_{B}
\end{align}
with \(G = \mathcal{H} \circ \mathcal{M}\)

The gradient of \(J\) is then
\begin{equation}
\nabla J(x) =
\nabla G(x)^TR^{-1}\left(G(x) - y\right) + B^{-1} (x - x_b)
\end{equation}
and the Hessian is
 \begin{equation}
\nabla^2 G(x) = \left(\nabla G(x)^T R^{-1} \nabla G(x) + B^{-1}\right) + Q(x)
\end{equation}
\subsection{Gauss-Newton method}
\label{sec:orgee34333}
Per, \href{../20220117085304-approximati_gn_methods_for_non_linear_least_square_problems.org}{Approximate GN methods for non-linear least-square problems} The
Gauss-Newton method relies on the knowledge of the GN-Hessian
\begin{equation}
\nabla G(x)^T R^{-1} \nabla G(x) + R^{-1}
\end{equation}


\subsection{AE formulation.}
\label{sec:org9efa493}
Let \(G = D \circ E \Rightarrow G(x) = D(E(x))\)
\begin{equation}
\nabla G(x) = \nabla D(E(x))\nabla E(x)
\end{equation}
So
\begin{align}
\nabla G(x)^T R^{-1} \nabla G(x) &=  \nabla E(x) ^T \nabla D(E(x))^T R^{-1}\nabla D(E(x))\nabla E(x)\\
&=  \nabla E(x) ^T \left(\nabla D(E(x))^T R^{-1}\nabla D(E(x))\right)\nabla E(x)
\end{align}

Let \(x \in \mathbb{R}^n\), \(G(x) \in \mathbb{R}^m\), and \(E(x) \in \mathbb{R}^{p}\)

Where \(n > m > p\)
\begin{align}
\nabla E(x)&: \mathbb{R}^n \rightarrow \mathbb{R}^{p \times n} \\
\nabla D(x)&: \mathbb{R}^p \rightarrow \mathbb{R}^{m \times p} \\
\nabla D(E(x)) &: \mathbb{R}^n \rightarrow \mathbb{R}^{m \times p}
\end{align}
So
\begin{equation}
\nabla G(x)^T R^{-1} \nabla G(x) = \underbrace{\nabla E(x)^T}_{\in \mathbb{R}^{n \times p}} \underbrace{\left(\nabla D(E(x))^T R^{-1}\nabla D(E(x))\right)}_{\in\mathbb{R}^{p \times p}}\underbrace{\nabla E(x)}_{\in\mathbb{R}^{p \times n}}
\end{equation}



Construct:
\begin{equation}
\nabla G(x^k) R^{-1} \nabla G(x^k)
\end{equation}


\bibliographystyle{apalike}
\bibliography{../../academic_writing/bibzotero}
\end{document}