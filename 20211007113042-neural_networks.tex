% Created 2021-10-08 ven. 19:13
% Intended LaTeX compiler: pdflatex
\documentclass[11pt]{article}
\usepackage[utf8]{inputenc}
\usepackage[T1]{fontenc}
\usepackage{graphicx}
\usepackage{longtable}
\usepackage{wrapfig}
\usepackage{rotating}
\usepackage[normalem]{ulem}
\usepackage{amsmath}
\usepackage{amssymb}
\usepackage{capt-of}
\usepackage{hyperref}
\author{Victor}
\date{\today}
\title{Neural Networks}
\hypersetup{
 pdfauthor={Victor},
 pdftitle={Neural Networks},
 pdfkeywords={},
 pdfsubject={},
 pdfcreator={Emacs 29.0.50 (Org mode 9.5)}, 
 pdflang={English}}
\begin{document}

\maketitle
\tableofcontents

\begin{verbatim}
import numpy as np
import matplotlib.pyplot as plt
import os
x = np.linspace(-10, 10, 100)
os.getcwd()
\end{verbatim}

\section{Neurons}
\label{sec:orgea582a2}

\subsection{Activation functions}
\label{sec:orgae47956}


\begin{verbatim}
def logit(x): return 1. / (1 + np.exp(-x))
plt.plot(x, logit(x))
plt.xlabel(r'$x$')
plt.ylabel(r'$\mathrm{logit}$(x)')
plt.title(r'Logit activation function')
plt.tight_layout()
fname = '/home/victor/org-roam/images/logit_activation.png'
plt.savefig(fname)
fname
\end{verbatim}

\begin{verbatim}
def tanh(x): return np.tanh(x)
plt.plot(x, tanh(x))
plt.xlabel(r'$x$')
plt.ylabel(r'$\tan h$(x)')
plt.title(r'Tanh activation function')
plt.tight_layout()
fname = '/home/victor/org-roam/images/tanh_activation.png'
plt.savefig(fname)
fname
\end{verbatim}
\end{document}